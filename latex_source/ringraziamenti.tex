\chapter*{Ringraziamenti}
\label{ringraziamenti}
\thispagestyle{empty}
\addcontentsline{toc}{chapter}{Ringraziamenti}

\section*{Ringraziamenti congiunti}
La riuscita di questo lavoro di tesi non è soltanto merito dei due autori, ma è il risultato della collaborazione e dell'impegno di molte persone.

Innanzitutto vorremmo ringraziare il nostro relatore, il professor \textit{Michele Norgia}, per aver sempre mostrato disponibilità, prontezza e gentilezza nel rispondere alle nostre domande ed ai nostri dubbi, anche ai più banali, e per aver contribuito a colmare le nostre lacune in un ambito abbastanza insolito per un ingegnere informatico.

Vorremmo inoltre ringraziare l'ormai Ingegner \textit{Samuele Disegna} per aver contribuito alla realizzazione fisica dello strumento descritto in questo lavoro e per aver dedicato il suo tempo libero, dopo la laurea, alla risoluzione dei problemi che si sono presentati durante la fase di test dello strumento.

Infine vorremmo ringraziare tutti i membri ed i tesisti del \textit{Laboratorio di Misure ottiche ed elettroniche (MOLES)}, per la disponibilità e la simpatia, che hanno contribuito a creare un ambiente di lavoro allegro ed amichevole. In particolare vorremmo ringraziare \textit{Giacomo}, che con la sua esperienza ci ha aiutato nell'ottimizzazione dello strumento, \textit{Federico}, per averci sempre spronato a fare di meglio ed il professor \textit{Alessandro Pesatori}, per averci fatto conoscere il mondo delle misure elettroniche.

\section*{Ringraziamenti individuali}
\subsection*{Diego ringrazia...}

Personalmente per prima cosa ritengo doveroso ringraziare la mia famiglia, a cui è dedicato questo lavoro, per avermi sempre sostenuto e supportato nei momenti difficili e per aver sempre creduto in me, anche nei momenti in cui non pensavo di riuscire a farcela.

Ringrazio i miei amici, \textit{Allo, Claudia, Manny} e \textit{Vale}, per aver sopportato in questi anni le mie battute senza senso e per avermi aiutato a distrarmi dallo studio e dai problemi con la loro compagnia.

Ringrazio \textit{Leonardo}, per la collaborazione mostrata nello sviluppo di questo lavoro e per l'impegno mostrato durante questi mesi.

Infine ringrazio i compagni di corso per aver contribuito ai miei risultati nei momenti di studio, ma soprattutto per aver reso meno pesante il tempo trascorso in università nei momenti di pausa.

\subsection*{Leonardo ringrazia...}

%% TODO Da sistemare Ringraziamenti Leo
Il ringraziamento più grande va alla mia \textit{famiglia} per il sostegno e la stima mostratomi in questi anni. \'E a loro che dedico questo lavoro di Tesi.

Ringrazio la mia fidanzata \textit{Sabrina}, per aver sempre creduto nelle mie potenzialità e per aver gioito con me dei miei successi e avermi consolato dopo le piccole sconfitte. 

Ringrazio il mio compagno di Tesi \textit{Diego}, per la collaborazione e per avermi sopportato e spronato nei momenti di sconforto durante lo svolgimento di questo lavoro.

Infine, ringrazio tutti gli \textit{amici} e i \textit{compagni di corso} per aver reso meno pesanti gli anni trascorsi in Università e per avermi aiutato a distrarmi dallo studio nei momenti di svago.

\subsection*{}
Senza l'aiuto, volontario o non, di queste persone non saremmo riusciti a raggiungere questo importante traguardo.