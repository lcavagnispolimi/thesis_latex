\chapter{Architettura Software dello strumento}
\label{capitolo4}
\thispagestyle{empty}

\textit{In questo Capitolo verrà descritta l’architettura software dello strumento realizzato. Inizialmente verrà analizzato l’ambiente di sviluppo software. Successivamente verrano mostrati i principi alla base della programmazione per microcontrollore e FPGA. Infine, verranno illustrate e descritte le funzionalità implementate su microcontrollore ed FPGA.}

\section{Ambiente di sviluppo LabVIEW}
%%TODO Cap 4
\section{Sistema Real-Time e FPGA}

	\subsection{Sistema Real-Time}

		\subsubsection{Sistema Operativo Real-Time}

			\paragraph{VxWorks}

		\subsubsection{LabVIEW Real-Time}

	\subsection{Progettazione e programmazione dell'FPGA}
		
		\subsubsection{Linguaggi di descrizione dell'Hardware (HDL)}
		
		\subsubsection{Sintesi Hardware}
		
		\subsubsection{High-Level Synthesis (HLS)}
		
		\subsubsection{LabVIEW FPGA}
		
\section{Analisi teorica degli algoritmi implementati}

	\subsection{Algoritmo di FFT}
		
	\subsection{Algoritmo di FFT Interpolata}
		
	\subsection{Misura di distanza}
	
\section{Architettura Software}

	\subsection{FPGA}
	
		\subsubsection{Implementazione software}
		
			\paragraph{Design pattern utilizzati}
		
				\subparagraph{Producer-consumer}
			
				\subparagraph{4-Wire Handshake}
			
			\paragraph{Fixed Point}
	
	\subsection{Microcontrollore}
	
		\subsubsection{Implementazione software}
	
	\subsection{Comunicazione tra FPGA e Microcontrollore}
		
		



